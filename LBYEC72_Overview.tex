\documentclass[compress,20]{beamer}
\mode<presentation>

\usetheme{Warsaw}
%\usefonttheme{default}
\usefonttheme{serif}
\hypersetup{pdfpagemode=FullScreen}

% define your own colours:
\definecolor{Red}{rgb}{1,0,0}
\definecolor{Blue}{rgb}{0,0,1}
\definecolor{Green}{rgb}{0,1,0}
\definecolor{magenta}{rgb}{1,0,.6}
\definecolor{lightblue}{rgb}{0,.5,1}
\definecolor{lightpurple}{rgb}{.6,.4,1}
\definecolor{gold}{rgb}{.6,.5,0}
\definecolor{orange}{rgb}{1,0.4,0}
\definecolor{hotpink}{rgb}{1,0,0.5}
\definecolor{newcolor2}{rgb}{.5,.3,.5}
\definecolor{newcolor}{rgb}{0,.3,1}
\definecolor{newcolor3}{rgb}{1,0,.35}
\definecolor{darkgreen1}{rgb}{0, .35, 0}
\definecolor{darkgreen2}{rgb}{0, .38, 0}
\definecolor{darkgreen}{rgb}{0, .6, 0}
\definecolor{darkred}{rgb}{.75,0,0}
\xdefinecolor{olive}{cmyk}{0.64,0,0.95,0.4}
\xdefinecolor{purpleish}{cmyk}{0.75,0.75,0,0}

\usecolortheme[named=darkgreen2]{structure} % Going Green

\useoutertheme{shadow}
\setbeamertemplate{background}[grid][step=0.5cm] %Masira ang sa taas
\usebeamertemplate*{logo}

% include packages
\usepackage{subfigure}
\usepackage{amstext}
\usepackage{latexsym}
\usepackage{epsfig}
\usepackage{graphicx}
\usepackage[all,knot]{xy}
\xyoption{arc}
\usepackage{url}
\usepackage{multimedia}
\usepackage{hyperref}
\usepackage{setspace}
%\usepackage{chess}
%\usepackage{graphpap}
%\usepackage{pstricks}
\usepackage{mathrsfs}
 

%My packages from mathbook
\usepackage{clock} %for clocks
\usepackage{ifsym} %  
\usepackage{epsdice} % dice
\usepackage{clock} % 
\usepackage{microtype}
\usepackage{skull} %create skull
\usepackage{cancel} %create diagonal bar (cancel)
\usepackage{stackrel} %create arc?
\usepackage{bbding} %create hands and cross
\usepackage{pifont} %create circled numbers
\usepackage{lipsum} % Just some sample text
\usepackage{amsmath,amssymb,amsthm} %math functions like align
\usepackage{fancybox} %For beautiful boxes
\usepackage{xspace} % Prints a trailing space in a smart way.
\usepackage{units}
\usepackage{geometry} % change paper size
\usepackage{multicol} % Small sections of multiple columns 
\usepackage{color}

%Animation
\usepackage{etex} % Solves ! No room for a new \dimen error
\usepackage{geometry}
\usepackage[T1]{fontenc}
\usepackage{lmodern}
\usepackage{tikz}
\usetikzlibrary{positioning,shadows,backgrounds}
\usepackage{animate}
\usepackage{calc}
\usepackage{color}
\usepackage{hyperref}
\usepackage{ifthen}



%My New Commands

\newcommand{\fig}[2]{
\begin{center}
\begin{figure}
\includegraphics[scale=#1]{figures/#2}
\end{figure}
\end{center}
}

%formula block
\newcommand{\f}[3]{
\begin{block}{\center  $ #1 $ }
\begin{center}
\begin{figure}
\includegraphics[scale=#2]{figures/#3}
\end{figure}
\end{center}
\end{block}
}

%question block
\newcommand{\q}[5]{
\begin{block}{#1 }
\begin{multicols}{2} \begin{enumerate}[]
\setlength\itemindent{0.69cm}
\item #2
\item #3
\item #4
\item #5
\end{enumerate}\end{multicols} 
\end{block}
}

\newcommand{\ans}[2]{\alert<#1>{\textbf<#1>{#2}}  \only<#1>{\textcolor {Red} \checkmark }} 


\newcommand{\dquote}[1]{\ding{125} \emph{#1} \ding{126}} %decorative quote

\newcommand{\formula}[1]{\vspace{-0.2cm}
\begin{center}
\shadowbox{#1}
\end{center}
\vspace{-0.2cm}
} % For formula


\newcommand{\point}[1]{\vspace{0.2cm} \HandCuffRight \, \smallcaps{#1}} %for pointing
\newcommand{\s}{\vspace{0.2cm}} % adds space
\newcommand{\ns}{\vspace{-0.5cm}}  % subtracts space
\newcommand{\g}[1]{\[\begin{gathered} #1 \end{gathered} \]}

%%% Steps %%%
\newcommand{\first}[1]{\ding{172} \smallcaps{#1}} 
\newcommand{\second}[1]{\ding{173}  \smallcaps{#1}} 
\newcommand{\third}[1]{\ding{174}  \smallcaps{#1}} 
\newcommand{\fourth}[1]{\ding{175}   \smallcaps{#1}} 


%%% EQUATIONS %%%
\newcommand{\one}{\quad \to \quad \text{\ding{182}}} 
\newcommand{\two}{\quad \to \quad \text{\ding{183}}} 
\newcommand{\three}{\quad \to \quad \text{\ding{184}}} 
\newcommand{\four}{\quad \to \quad \text{\ding{185}}} 
\newcommand{\five}{\quad \to \quad \text{\ding{186}}} 
\newcommand{\six}{\quad \to \quad \text{\ding{187}}} 
\newcommand{\seven}{\quad \to \quad \text{\ding{188}}} 
\newcommand{\eight}{\quad \to \quad \text{\ding{189}}} 


%My new environments

%New environment for choices in two columns %Modify to single column (4/14)
\newenvironment{choicescol}{\vspace{-0.12cm}
\begin{itemize}[]\setlength\itemindent{0.12cm} }
{\end{itemize} 
\vspace{0.2cm}
}

%New environment for choices in two columns
\newenvironment{choices}{\vspace{-0.4cm}
\begin{multicols}{2} \begin{itemize}[]\setlength\itemindent{0.69cm} }
{\end{itemize}\end{multicols}} 


 

\title{$\mathcal{LBYEC}72$}
\subtitle{Computer Fundamentals : Programming 2 \\
Pre-requisite: LBYEC71(Soft)
}
\author{ Engr. Melvin Kong Cabatuan}
\institute{{\Large De La Salle University }  \\ 
Manila, Philippines \\
}

\date{\small January 2013 \ns }



\begin{document}
\begin{LARGE}
 



\frame{

\titlepage

\begin{center}
\begin{figure}
\includegraphics[scale=0.2]{figures/dlsulogo}
\end{figure}
\end{center}

}


\logo{\includegraphics[height=2cm]{figures/dlsulogo}}


\frame{ \frametitle{Self Introduction}
\ns
\fig{0.45}{me2}
\ns
 \noindent\underline{\textsc{Melvin K. Cabatuan, MsE, Ph.D. Cand.}} 
 \begin{itemize}
\normalsize
\centering
\item Masters of Engineering, NAIST (Japan)
\item Thesis: Cognitive Radio (Wireless Communication)
\item ECE Reviewer/Mentor (Since 2005)
\item 2nd Place, Nov. 2004 ECE Board Exam
\item Test Engineering Cadet, ON Semiconductors
\item DOST Academic Excellence Awardee 2004  
\item Mathematician of the Year 2003
\item DOST Scholar (1999-2004)
\item Panasonic Scholar, Japan (2007-2010)  
\end{itemize} 
\vfill
}


\begin{frame}
\frametitle{On Doing Research}
\fig{0.6}{mthesis}
\end{frame}


\section{Introduction}

\frame{\tableofcontents}


\section{Course Contents}

\frame{\frametitle{Course Contents }
\setbeamercovered{transparent}
\begin{itemize}
\Large
\item<1->  Review of Conditional and Iterative Statements, Arrays, and Strings
\item<2->  Topic 1: Nested Conditional and Iterative Statements
\item<3->  Topic 2: Single-Dimensional and Multi-dimensional Arrays 
\item<4->  Topic 3: Strings, String Arrays, and String Manipulation Functions
\item<5-> \textbf{ Practical Exam 1}
\item<6->  Discussion on Pointers, Functions, and Structures
\end{itemize}
\vfill
}
 
\frame{\frametitle{Course Contents }
\setbeamercovered{transparent}
\begin{itemize}
\Large
\item<1->  Topic 4: Pointers
\item<2->  Topic 5: Functions and Pass-by-value
\item<3->  Topic 6: Functions and Pass-by-reference
\item<4->  Topic 7: Structures, Structure Array, and Complex Data Type
\item<5->  Topic 8: Structures, Structure Pointers, and Passing of References
\item<6->  \textbf{Practical Exam 2}
\item<7->  Discussion on Dynamic Memory \\ Allocation and Exercise
\end{itemize}
\vfill
}
 


\subsection{References}
\frame{\frametitle{References}
\setbeamercovered{transparent}
\begin{enumerate}
\Large
\item<1-> LBYEC72 Laboratory Manual
\item<2-> Books and other online sources
\end{enumerate}
\vfill
}

 
\subsection{Evaluation}
\frame{\frametitle{Evaluation Criteria}
\setbeamercovered{transparent}
\begin{table}
  \centering
  \begin{tabular}{lc}
    Average of Preliminary Reports: & 20\%  \\
    Average of Final Reports: & 20\%  \\
    Project: &  30\%   \\
    Practical Examination I : & 15\%   \\
    Practical Examination II : & 15\% \s  \\ 
       \hline \\ 
    Total: & 100\% \\
    PASSING GRADE:  \: 70\% & {}\\
    \end{tabular}
\end{table}
\vfill
}


\subsection{Prelim}

\frame{\frametitle{Preliminary Report}
\begin{enumerate}
\item Preliminary Reports are written and completed prior to the end of every laboratory sessions using your EC72 journal.
\item  Preliminary Reports are checked 30 minutes before the end of every session.
\item Preliminary Reports are individual.
\end{enumerate}
 }


\subsection{Final}

\frame{\frametitle{Final Report}
\begin{enumerate}
\item Final Reports should be submitted one week after the topic.
\item Late reports will receive a 10 \% deduction per week.
\item Final Reports are done by pair.
\end{enumerate}
 }

\subsection{Project}

\frame{\frametitle{Project}
\begin{enumerate}
\item Students may develop a project proposal or follow the project specifications given by the instructor.
 \item Projects are done by groups with a maximum of three members.
\end{enumerate}
 }
 



\section{Programming Review}

\frame{\frametitle{Programming Review: Hello World!}
 }


 
\frame{\frametitle{Programming Review}
\begin{block}{Problem 1}
 Given the quadratic equation $ax^2+ bx+c=0$. Write a simple program that implements the following quadratic formula: \\
\[ x = \dfrac{-b \pm \sqrt{b^2 - 4ac}  }{2a} \]
\s
\end{block}
}


\frame{\frametitle{Sample Answer:}
 \fig{0.56}{quadratic}
\vfill
 }


\frame{\frametitle{Programming Review}
\begin{block}{Problem 2}
Write a program that prints the maximum of four given integers.
 \fig{0.9}{maxproblem}
\end{block}
}


\frame{\frametitle{Sample Answer:}
 \fig{0.8}{maxproblemans}
\vfill
 }


\frame{\frametitle{Programming Review}
\begin{block}{Problem 3}
Write a program that prints a tringle of stars shown in the following figure:
 \fig{0.9}{triangle}
\end{block}
}
 
\frame{\frametitle{Sample Answer:}
 \fig{0.8}{triangleans}
\vfill
 }


\frame{\frametitle{Programming Review}
\begin{block}{Problem 4}
Write a program that prints a diamond of stars shown in the following figure:
 \fig{0.9}{diamond}
\end{block}
}


\frame{\frametitle{Sample Answer:}
 \fig{0.7}{diamondans}
\vfill
 }

\frame{\frametitle{END}
\begin{center}
\dquote{Thank you for your attention}
\fig{0.45}{gundam}
\end{center} 
}

\end{LARGE}
\end{document}  
